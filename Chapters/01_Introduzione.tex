\chapter{Introduzione}
Per dare una valutazione di incertezza \`e necessario definire il metodo dell'esperimento.
\section{Esperimento casuale o esperimento aleatorio}
Un esperimento determinato da un certo individuo in un certo istante. Pur avendo una serie di informazioni su di esso l'individuo non \`e in grado di determinarne con certezza
il risultato, indipendentemente dal fatto che l'esperimento sia gi\`a avvenuto oppure no.
\subsection{Spazio campionario}
Lo spazio che contiene tutti risultati possibili dell'esperimento. Indicato con $\Omega$. Gli elementi di $\Omega$ sono mutualmente esclusivi (l'avvenimento di uno esclude gli altri) e 
lo spazio \`e esaustivo (contiene tutti i risultati possibili). Gli elementi di $\Omega$ si chiamano eventi elementari. 
\section{Insiemistica}
Si definisca:
\begin{itemize}
\item $A^c$ il complementare di $A$.
\item $A\backslash B=A\cap B^c$ la differenza tra due insiemi.
\item $A\Delta B=(A\backslash B)\cup(B\backslash A)$.
\end{itemize}
\subsection{Propriet\`a unione e intersezione}
\begin{itemize}
\item Idempotenza: $A\cap A=A$ e $A\cup A=A$
\item Commutativit\`a.
\item Associativit\`a. 
\item Distributivit\`a: $A\cup(B\cap C)=(A\cup B)\cap(A\cup C)$
\item $A\cap\emptyset=\emptyset$, $A\cup\emptyset=A$, $A\cap\Omega=A$, $A\cap\Omega=\Omega$
\end{itemize}
\subsection{Leggi di De Morgan} 
\begin{itemize}
\item $(A\cup B)^c=A^c\cap B^c$
\item $(A\cap B)^c=A^c\cup B^c$
\end{itemize}
\subsection{Operazioni per gruppi di insiemi}
Si supponga di avere un numero finito di insiemi $A_1, A_2, \cdots, A_n$.
\begin{itemize}
\item L'unione viene indicata con $\bigcup\limits_{i=1}^nA_i$.
\item  l'intersezione: $\bigcap\limits_{i=1}^nA_i$
\end{itemize}
\subsubsection{Regole di De Morgan} 
\begin{itemize}
\item $(\bigcup\limits_{i=1}^nA_i)^c=\bigcap\limits_{i=1}^nA_i^c$
\item $(\bigcap\limits_{i=1}^nA_i)^c=\bigcup\limits_{i=1}^nA_i^c$
\end{itemize}
\subsection{Successioni di insiemi}
Le successioni di insiemi sono numerabili, contengono ovvero elementi pari al pi\`u ai numeri naturali. Ovvero esiste una funzione biettiva tra i numeri naturali e gli insiemi.Si 
possono allora considerare:
\begin{itemize}
\item Unione $\bigcup\limits_{i=1}^\infty A_i$
\item Intersezione $\bigcap\limits_{i=1}^\infty A_i$
\end{itemize}
Considerando una successione pi\`u che numerabile si descrive $\bigcap\limits_{\alpha\in\Lambda}A_\alpha$, dove $\Lambda$ \`e un insieme maggiore dei numeri naturali.
\subsection{Considerazioni su sottoinsiemi}
Siano $A,B\subseteq\Omega$, se $A\subseteq B$ $A\cap B=A$:
\begin{itemize}
\item $A\cup B=B$
\item $B^c=\Omega\backslash B\subseteq \Omega\backslash A=A^c$ 
\item $A\cap B^c=\emptyset$
\item $A^c\cap B=\Omega$
\end{itemize}
\section{Gli eventi}
I potenziali eventi sono sottoinsiemi dell'insieme spazio campionario. Il numero di eventi \`e dato dalla cardinalit\`a dell'insieme delle parti dello spazio campionario:
$2^{\#\Omega}$. Questo insieme viene detto insieme potenza, indicato con $P(\Omega)$ e contiene tutti i sottoinsiemi di $\Omega$. Gli eventi sono gli elementi dell'insieme
potenza.
\subsection{Propriet\`a}
Una classe $\mathcal{A}$ di parti di un insieme $\Omega$ ($\mathcal{A}\subseteq P(\Omega)$) \`e detta trib\`u (o $\sigma$-algebra) se:
\begin{itemize}
\item $\Omega\in \mathcal{A}$
\item Deve essere chiusa rispetto all'operazione complementare se $A\in \mathcal{A}$ allora $A^c\in \mathcal{A}$.
\item $\{A_i\}_{i=1}^{+\infty}$ tale che $A_i\in \mathcal{A}\Rightarrow \bigcup\limits_{i=1}^{+\infty}A_i\in \mathcal{A}$ (metto in successione infinita ogni combinazione di elementi di $\mathcal{A}$).
\end{itemize}
Un algebra \`e una $\sigma$-algebra la cui terza propriet\`a cambia in: $\{A_i\}_{i=1}^{n}$ tale che $A_i\in \mathcal{A}\Rightarrow \bigcup\limits_{i=1}^nA_i\in \mathcal{A}$. Una $\sigma$-algebra
\`e sempre un'algebra. La pi\`u piccola trib\`u \`e $\mathcal{A_1}=\{\emptyset,\Omega\}$. Una trib\`u generata da un'insieme di partenza \`e detta trib\`u generata da tale insieme. 
\subsection{Definizione}
Un evento \`e un elemento della trib\`u che viene utilizzata. 
