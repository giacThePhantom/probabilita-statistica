\chapter{Probabilit\`a condizionale}
\subsubsection{Definizione}
Sia $(\Omega,\mathcal{A},Pr)$ uno spazio probabilizzato. Fissato un elemento $H$ di $\mathcal{A}$, con $Pr(H)\neq 0$, si chiama funzione di probabilit\`a dedotta 
da $Pr$ sotto la condizione $H$ la funzione di probabilit\`a $Pr_H$ sullo spazio $(\Omega,\mathcal{A})$ probabilizzabile:
\begin{equation}
Pr_H(A)-\dfrac{Pr(A\cap H)}{Pr(H)}
\end{equation}
Per ogni evento $A\in\mathcal{A}$. La probabilit\`a $Pr_H(A)$ si chiama probabilit\`a condizionale di $A$ secondo $Pr$ sotto la condizione $H$ e si denota $Pr(A|H)
$.
\section{Spazio di probabilit\`a condizionale}
Dopo che \`e accaduto l'evento $H$ per ogni evento $A\in\mathcal{A}$ tale che $A\cap H=\emptyset$ si ha che $Pr_H(A)=PR(A|H)=0$, avremo quindi un altro modo di 
definire lo spazio probabilizzato $(\Omega_H, \mathcal{A}_H, Pr_h)$ dove $\Omega_H=\{A\subseteq H:A\in\Omega\}$ e poich\`e accadono solo gli eventi che implicano 
$H$, $\mathcal{A}_H=\{A\subseteq H:A\in\mathcal{A}\}$. 
\section{teorema delle probabilit\`a totali}
\subsection{Enuncitato}
Sia $\{A_i\}_{i=1}^\infty$ una famiglia di eventi che costituisce una classe completa su $\Omega$ tale che $Pr(A_i)>0\;\;\forall i$ e sia $B$ un qualunque evento, 
allora:
\begin{equation}
Pr(B)=\sum\limits_{i=1}^\infty Pr(A_i\cap B)=\sum\limits_{i=1}^\infty Pr(A_i)Pr(B|A_i)
\end{equation}
\subsubsection{Dimostrazione}
Dalla relazione $\Omega=\bigcup\limits_{i=1}^\infty A_i$, intersecando con $B$ ambo i membri e per la propriet\`a distributiva rispetto all'unione si ottiene:
$\Omega\cap B=\bigcup\limits_{i=1}^\infty(A_i)\cap B=\bigcup\limits_{i=1}^\infty(A_i\cap B)$, con gli eventi $A_i\cap B$ incompatibili a due a due. Segue allora che
$Pr(B)=Pr(\bigcup\limits_{i=1}^\infty(A_i\cap B))=\sum\limits_{i=1}^\infty Pr(A_i\cap B)=\sum\limits_{i=1}^\infty Pr(A_i)Pr(B|A_i)$.
\section{Teorema di Bayes}
Sia $\{A_i\}_{i=1}^\infty$ una famiglia di eventi che costituisce una classe completa su $\Omega$ tale che $Pr(A_i)>0\;\;\forall i$ e sia $B$ un qualunque evento, 
allora:
\begin{equation}
Pr(A_i|B)=\dfrac{PR(A_i)Pr(B|A_i)}{\sum\limits_{j=1}^\infty Pr(A_j)Pr(B|A_j)}
\end{equation}
\subsubsection{Dimostrazione}
La dimostrazione \`e da fare, da ricordarsi, immediata dal teorema delle probabilit\`a totali.
