\chapter{Valori notevoli di variabili aleatorie}
\section{Valore atteso}
Un valore atteso \`e una sintesi della variabile aleatoria in un numero: $\mathbb{E}:(\mathbb{R}, \mathcal{B}(\mathbb{R}), F_x)\rightarrow\mathbb{R}$. Il valore atteso 
rappresenta la media pesata di tale variabile aleatoria.
\subsubsection{Valore atteso per variabili aleatorie discrete}
Si supponga $X$ una variabile aleatoria discreta con funzione di probabilit\`a $p_x(x)$, si chiamer\`a $\mathbb{E}(X)$ la quantit\`a $\mathbb{E}(X)=\sum\limits_{x\in R_x}
xp_x(x)$ se $\sum\limits_{x\in R_x}|x|p_x(x)<\infty$.
\subsubsection{Valore atteso per variabili aleatorie continue}
Si supponga $X$ una variabile aleatoria continua con funzione di densit\`a $p_x(x)$, si chiamer\`a $\mathbb{E}(X)$ la quantit\`a $\mathbb{E}(X)=\int_{-\infty}^{+\infty}xp_x(x)$ 
se $\int_{-\infty}^{+\infty}|x|p_x(x)<\infty$.
\section{Trasformazioni di variabili aleatorie}
Sia $X$ una variabile aleatoria e si definisca $Y=a+bX$, $\mathbb{E}(Y)=\mathbb{E}(a+bX)=\sum\limits_{x\in R_X}(a+bx)p_x(x)=\sum\limits_{x\in R_X}(ap_x(x)+bxp_x(x))=\sum
\limits_{x\in R_X}ap_x(x)+\sum\limits_{x\in R_x}bxp_x(x)=a\sum\limits_{x\in R_X}p_x(x)+b\sum\limits_{x\in R_x}xp_x(x)=a+b\mathbb{E}(X)$. Il valore atteso di una trasformazione 
lineare di una variabile aleatoria \`e la trasformazione lineare del valore atteso. Pi\`u in generale il valore atteso di $h(Y)$, dove $Y$ \`e una variabile aleatoria discreta 
pu\`o essere espresso come $\mathbb{E}(h(Y))=\sum\limits_{i=1}^kh(y_i)p(y_i)$. Analogamente con l'integrale per le variabili aleatorie continue.
\section{Momenti}
\subsection{Momenti non centrati}
Sono valori attesi $\mathbb{E}(X^r)$ dove $r$ (deciso arbitrariamente, fissato) \`e un intero positivo, chiamato ordine del momento. $\mathbb{E}(X^r)=\sum\limits_{x^r\in R_x}
x^rp_x(x)$.
\subsection{Momenti centrati}
Si definisce il momento centrato come $\mathbb{E}((X-\mathbb{E}(X))^r)$.
\subsection{Varianza}
$\sigma^2=\mathbb{E}((X-E(X))^2)$ \`e la varianza della variabile aleatoria. Il valore atteso \`e nella stessa unit\`a di misura della variabile aleatoria. \`E una misura di 
dispersione, dice quanto si discosta il valore dal valore atteso. 
\subsubsection{Deviazione standart}
Si definisce la deviazione standard come la radice della varianza $\sigma=\sqrt{\mathbb{E}((X-E(X))^2)}$.
\subsection{Coefficiente di variazione}
In modo analogo alla varianza indica la dispersione ma indipendentemente dall'unit\`a di misura. $CV=\frac{\sqrt{\mathbb{E}((X-E(X))^2)}}{|\mathbb{E}(X)|}$. 
\subsection{Momento terzo centrato}
$\rho_3=\frac{\mathbb{E}((X-E(X))^3)}{\sigma^3}$. Se la variabile \`e simmetrica rispetto al suo valore atteso il numeratore \`e zero, che \`e il valore minore del momento 
terzo, pertanto valori piccoli quando la variabile tende verso la simmetria. L'implicazione al contrario non \`e vera.
\subsection{Momento quarto centrato}
$\rho_4=\frac{\mathbb{E}((X-E(X))^4)}{\sigma^4}$ , indice di Curtosi, ovvero la misura di quanto la massa \`e concentrata la media e di quanto velocemente vanno a zero le code, 
indici grandi code che scendono a zero lentamente e appuntimento in centro. La curtosi per la normale \`e tre. 
\section{Variabile aleatoria degenere}
La variabile aleatoria degenere assume valore di probabilit\`a $1$ in un punto $c$. Il valore atteso di una costante \`e la costante stessa. Si consideri la variabile aleatoria
$Y$ costruita come $Y=cg(X)$, allora se $c=1$, $Y=g(X)$, $R_Y$, il supporto della nuova variabile aleatoria, e si calcoli $Pr(Y_j)$, $\mathbb{E}(Y)=\sum\limits_{i\in R_Y}yPr(y)=
\sum\limits_{j=1}^J y_jPr(y_j)=\sum\limits_{j=1}^J y_jP_x(g^{-1}(y_j)) $. $Y=g(x)$, $\mathbb{E}(X)=\mathbb{E}(g(X))$. Se $c\neq 1$, si definisca $h(x)=cg(x)$, $\mathbb{E}(Y)=
\sum\limits_{x\in R_X} h(x)p_x(x)=\sum\limits_{x\in R_X} cg(x)p_x(x)=c\sum\limits_{x\in R_X} g(x)p_x(x)=x\mathbb{E}(g(X))$.