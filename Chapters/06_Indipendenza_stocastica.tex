\chapter{Indipendenza stocastica}
\section{Indipendenza di due eventi}
Due eventi si dicono stocasticamente indipendenti quando l'avvenimento di uno non modifica la misura di incertezza dell'altro.
\subsubsection{Definizione}
In uno spazio probabilizzato $(\Omega,\mathcal{A},Pr)$ due eventi $A$, $B$ si dicono stocasticamente indipendenti se e solo se $Pr(A\cap B)=Pr(A)Pr(B)$, si noti
che in particolare $Pr(A|B)=Pr(A)$ e $Pr(B|A)=Pr(B)$.
\section{Indipendenza di n eventi}
La notazione di indipendenza stocastica pu\`o essere 
\subsubsection{Definizione}
Si consideri $(\Omega,\mathcal{A},Pr)$ e $\{A_i\}_{i=1}^n:A_i\in \mathcal{A}\;\;\forall i$ tali eventi sno stra loro stocasticamente indipendenti se e solo se 
$Pr(A_{i_1}\cap A_{i_2}\cap A_{i_k})=Pr(A_{i_1})Pr(A_{i_2})Pr(A_{i_k})$, $\forall k=2,\cdots,n$ e per ogni allineamento $(i_1<i_2<\cdots<i_k)$ degli interi $1,2,
\cdots, n$, ovvero gli eventi di una famiglia si dicono mutualmente indipendenti se ogni ennupla da essa estratta risulta stocasticamente indipendente.
\section{Trib\`u indipendenti}
Dato uno spazio probabilizzato $(\Omega,\mathcal{A},Pr)$ due trib\`u contenute in $\mathcal{A}$ si dicono tra loro stocasticamente indipendenti se ogni elemento 
dell'uno \`e indipendente da ogni elemento dell'altra.